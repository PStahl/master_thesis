\documentclass{article}
\usepackage[utf8]{inputenc}
\usepackage[swedish]{babel}

\usepackage{multicol}
\usepackage[textwidth=440pt]{geometry}


\author{
	Philip Ståhl \& Jonatan Broberg
}

\title{Ökad tillförlitlighet för molntjänster}
\date{\today}

\pagenumbering{gobble}
\hyphenation{Ericsson}

\begin{document}
\maketitle

\begin{multicols}{2}
%Preamble
\noindent
\textbf{Med automatiskt feldetektering och periodisk övervakning av molnapplikationer kan en önskad nivå av tillförlitlighet uppnås genom automatiskt kloning av applikationen. Nackdelen är att fler resurser krävs och därför måste antalet kopior hållas till ett minimum genom att inte ha fler kopior än nödvändigt.}

\section*{Metod}
Genom att skapa identiska kopior av funktioner eller beräkningar som en tjänst utför, så ökar sannolikheten att minst en av dessa lyckas slutföra beräkningen utan att ett fel inträffar. Genom att skapa tillräckligt många kopior kan en specifik tillförlitlighetsnivå uppfyllas, d.v.s. sannolikhet att minst en kopia lyckas slutföra sin beräkning och producera ett resultat. Vidare kan den önskade tillförlitlighetsnivån uppnås över en längre tid genom att automatiskt skapa nya kopior allt eftersom tidigare kopior slutar fungera till följd av serverfel.

Med flera identiska kopior som utför exakt samma uppgift, så ökar tillförlitligheten, till följd av att fler resurser krävs i form av beräkningskraft och energiförbrukning. Därför är det viktigt att hålla antalet kopior till ett minimum, så att inte onödigt mycket resurser används. Dessutom är det svårt att tillgodose en viss tillförlitlighet för molntjänster eftersom risken för fel varierar över tid. Exempelvis är risken för serverfel högre om servern är under hög belastning. Eftersom tillförlitligheten varierar över tid, så måste applikationer och tillgängliga servrar periodiskt övervakas för att se till att den önskade nivån uppfylls. Genom att flytta kopior till mer tillförlitliga servrar kan den önskade nivån uppnås med färre antal kopior. Överflödiga kopior kan då tas bort och därmed resursåtgången hållas till ett minimum.

\section*{Bakgrund}
Intresset för applikationer och tjänster som kör i molnet har vuxit explosionsartat det senaste decenniet. Molnet är en bred term och används för att beskriva distribuerade kluster av servrar på vilka applikation och tjänster körs. Dessa servrar är oftast byggda med relativt billiga komponenter med hög felfrekvens. Fel som leder till att en server blir oanvändbar inträffar ofta, och sådana fel är ofta svåra att förutse. Om ett fel inträffar, så kan inte längre de beräkningar som utfördes på servern slutföras, och värdefulla resultat kan gå förlorade. Vidare så kräver användare av sådana tjänster oftast en viss tillförlitlighet, d.v.s. sannolikhet att en applikation eller tjänst lyckas utföra det jobb den är ämnad för utan att fel inträffar. 

\section*{Resultat}
Inom forskningsvärlden har det tidigare saknats en plattform där man kunnat utföra realistiska experiment på ett enkelt sätt. Ovannämnda funktionalitet för att automatiskt skapa nya kopior, och detektera fel, har implementerats i \emph{Calvin}, en applikationsmiljö för Internet of Things utvecklad av Ericsson. Därmed finns nu en plattform tillgänglig där nya metoder kan prövas och experiment utföras.

\end{multicols}
\end{document}


\iffalse
Guide: http://www.student.lth.se/fileadmin/lth/anstallda/kvalitet/examensarbete/Guide_foer_populaervetenskapligt_skrivande_141015.pdf

Rubriken bör vara kort och kärnfull och fungera som en ”supersammanfattning” av vad ditt exjobb handlar om. Inte längre en rad/ca 100 tecken. 

Ingressen ska med ett par få meningar locka till vidare läsning och/eller sammanfatta det viktigaste av ditt arbete på ett intresseväckande sätt. Observera att det är rubriken samt ingressens inledning, runt 35 ord/200 tecken, som kommer att exponeras via webb feeden. Lägg därför ner lite extra arbete på dessa stycken! 

Brödtexten är huvuddelen av artikeln. Inte mer än en A4, max 3000 tecken. 

En lämplig målgrupp för en populärvetenskaplig sammanfattning kan vara de studenter som går första årskursen på en civilingenjörsutbildning

http://cs.lth.se/instruktioner-old/popvet/ 


ATT BESVARA:
Vad har du arbetat med? Har du kommit fram till något intressant? (Resultat) 
 - arbetat med att tillgodose en önskad tillförlitlighet genom att skapa identiska kopior av funktioner och uppgifter

Vilka behov/problem adresserar du i ditt exjobb? (Behov)  
 - undvika att värdefull data går förlorad till följd av fel tillgodose en tillförlitlighet som användaren önskar på ett “seamless” sätt

Varför är det relevant att komma till rätta med dessa problem/behov? (Nytta) 
 - användare kräver att de tjänster de använder fungerar med väldigt hög sannolikhet - vilket är svårt att garantera

Hur kan ditt arbete komma till användning? (Tillämpning och konsekvens) 
 - forskningsvärlden har fått en plattform att experimentera på

\fi